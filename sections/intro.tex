\section{Introduktion till projektet}
Detta projekt har genomförts inom ramen för kursen Produktutveckling i Medieteknik vid Civilingenjörsprogrammet i Interaktion och Design, i folkmun även kallad DBT-kursen (Design-Build-Test). I kursen fungerar vi studenter som konsulter och arbetar i team med verkliga uppdragsgivare. Syftet med kursen är att träna oss i att hantera hela utvecklingsprocessen med hjälp av DBT-metoden. Från att identifiera behov, till att designa, bygga och testa en lösning. Vårt team, Team 5, består av fyra konsulter: Angelica Nordin, Elias Kristmansson, Ludwig Lönnberg och Tim Sjöström.

\subsection{Bakgrund}
Elias Kristmansson har en personlig koppling till en av ägarna i det Umeå-baserade företaget Pelagia AB. Denna koppling ligger till grund för hur samarbetet med dem uppstod. Det är även tack vare denna koppling som kommunikationen mellan projektgruppen och företaget fungerat mycket smidigt under hela projektets gång. Detta har i sin tur skapat goda förutsättningar för att förstå företagets specifika behov och ta fram relevanta lösningar. Arbetsgivaren Pelagia Nature and Entertainment AB arbetar med att analysera biologiskt material ifrån våtmarker, sötvatten och saltvatten. Pelagia tar emot och analyserar dessa biologiska prover från kunder runt om i Norden. Vid mottagning och analys sker dokumentationen av provernas innehåll vid olika stationer på Pelagias kontor. Denna process utförs idag i långa Excel ark, där varje arbetsstation har sitt eget typ av dokument, något som gör det svårt att få en samlad översikt över informationen. Det manuella arbetssättet bidrar också till att processen tar längre tid än nödvändigt, samt att risken för fel eller inkonsekvent datahantering ökar. Det är här vi kommer in.

\subsection{Vårt uppdrag}
Vårt uppdrag är att skapa ett mer intuitivt och användarvänligt gränssnitt för dokumentation av prover som är applicerbart vid varje arbetsstation. Gränssnittet ska förbättra översikten över insamlad data, göra det enklare att skriva in och sortera information samt minska inlärningstiden för nya användare. Målet med projektet är inte att ändra i Pelagias nuvarande arbetsprocess utan istället att ge dem ett mer rigoröst och användarvänligt verktyg att använda sig utav vid dokumentation av prover. Eftersom projektet genomförs inom ramarna för vår designkurs så kommer vi använda oss av agila arbetsmetoder där fokus ligger på att designa, bygga och testa lösningar i korta iterationer tillsammans med Pelagia. Vid kursens slut så kommer vi överlämna koncept och idé till Pelagia vilket dem sedan kan vidareutveckla och implementera. 