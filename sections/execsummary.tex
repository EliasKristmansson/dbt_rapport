\section{Executive Summary}
Projekt Streamline är ett projekt som utgör den huvudsakliga delen av kursen Produktutveckling i medieteknik med metoden 'Design-Build-Test' vid Umeå Universitet. Det innefattar att en projektgrupp ska driva ett värdeskapande projekt tillsammans med ett företag med metoden 'Design-Build-Test'. Metoden är ett tillvägagångssätt inom design som handlar om att designa något, bygga detta och sedan testa det som byggts. Det är en iterativ process vilket innebär att kretsloppet med 'Design-Build-Test' sker i flera cyklar tills ett bra resultat har uppnåtts.
\\
\\
Projekt Streamline har drivits tillsammans med företaget Pelagia Nature and Environment AB. Pelagia är ett företag från Umeå som arbetar med utredningar av olika naturmiljöer, särskilt angående vatten och djurliv, med syfte att hjälpa företag ta hänsyn till miljön vid byggprojekt eller liknande. De har ett laboratorium i Umeå där de utför biologiska analyser av prover från olika naturområden där de främst fokuserar på att plocka ut små djur för att bestämma mängd och deras arter.
\\
\\
Projektet har drivits över en period på dryga 5 månader med fokus på agila projektledningsmetoder. Det är Scrum som huvudsakligen har använts som mall i projektledningen men det har inte gått att följa till punkt och pricka då kursen projektet är en del av är på 25\%. Detta har lett till längre perioder på 1-2 veckor då nästintill inget arbete har utförts, vilket inte stämmer bra överens med Scrum då det egentligen ska innebära kontinuerligt och regelbundet arbete och avstämningar. Det är fyra medlemmar med i projektgruppen och alla har haft sin roll i projektet. Majoriteten av utvecklingen har dock skett parallellt där alla medlemmar har suttit tillsammans och programmerat. Detta ledde till att det interna samarbetet i projektgruppen blev smidigare.
\\
\\
Det absolut första att göra i projektet (till och med innan det blev tilldelat namnet Projekt Streamline), var att hitta rätt företag att arbeta med. På en eftermiddag hade ett dokument med ett tiotal företag tagits fram. Bland dessa fanns de två företag som blev de slutgiltiga kandidaterna för projektet, NorrSpect och Pelagia. Efter möten med båda företagen med diskussioner kring vad som rimligtvis skulle kunna göras, togs beslutet att Pelagia skulle bli företaget att samarbeta med. 
\\
\\
Med ett företag valt var nästa steg att skapa ett projektdirektiv med information om vad projektet skulle innebära. Idén var att hjälpa Pelagia att ta fram ett nytt dokumentationssystem. Det är nämligen så att Pelagia i nuläget använder Excel för att skriva in datan från deras biologiska analyser. Detta ansåg Pelagia vara någonting som, inom en snar framtid, skulle ställa till problem eftersom företaget växer och eftersom Excel inte riktigt var lämpligt för just deras situation. De argumenten som togs upp var att kalkylbladen som Pelagia använder idag är alldeles för stora och ostrukturerade för den mängd data som skrivs in och antalet personer som måste kunna redigera kalkylbladen samtidigt. Det är alltså kalkylblad sorterade efter år med tusentals rader per blad och dessutom en stor mängd kolumner. Pelagia kände att det var svårt att hitta rätt i arken och jobbigt att behöva scrolla långt åt höger för att komma åt precis de kolumner de letar efter.
\\
\\
Strax därefter kunde en kravspecifikation och projektplan med mer konkreta milstolpar tas fram. Dessa dokument var till mer nytta för projektgruppen än för Pelagia eftersom de endast är menade att strukturera själva projektet och hur det ska drivas. I dessa finns tidsplanen för projektet, beslutspunkter, en enklare mötesplan, dokumentplan, organisationsplan och ytterligare punkter för att ge projektgruppen en bra utgångspunkt för det kommande utvecklingssteget.
\\
\\
Nästa del i projektet var att samla in data om hur Pelagia faktiskt ville att systemet skulle se ut. För detta hölls 6 kvalitativa intervjuer på plats hos Pelagia med personer som dagligen interagerar med det nuvarande systemet. Det som genomsyrade intervjusvaren var att systemet behövde bli mer effektivt, lättlärt och det absolut viktigaste, att det skulle bli lättare att få en översikt över systemet. Med översikt inkluderas då sätt att söka, sortera eller filtrera prover och projekt för att enklare kunna hitta i systemet. En annan sak som togs upp ofta var att utseende är relativt ointressant, utan att funktionalitet var det absolut viktigaste.
\\
\\
Med denna information kunde den första designiterationen påbörjas. Det första steget var att skapa ett antal prototyper. För detta användes Figma. I Figma togs tre prototyper fram med varierande grad av komplexitet, en lofi-prototyp, en hifi-prototyp och en som var något mittemellan. Lofi-prototypen var väldigt enkel både designmässigt och funktionsmässigt. Hifi-prototypen hade bra design och använde även klick-funktionalitet för att visa hur interaktion med den slutgiltiga produkten hade kunnat se ut. Prototypen mittemellan var väl designad men hade ingen klick-funktionalitet. Dessa prototyper visades sedan upp hos Pelagia under en kortare presentation för att samla in feedback för den första iterationen.
\\
\\
Redan innan prototyperna togs fram bestämdes det att själva programmeringen skulle ske i React utan någon kopplad databas, då detta ansågs vara för tidskrävande och eftersom vissa medlemmar i projektgruppen inte hade någon tidigare erfarenhet av databasprogrammering. Med feedback från presentationen hos Pelagia kunde projektgruppen börja programmera webbplatsen. Under detta steg stagnerade kommunikationen med Pelagia något eftersom det inte fanns mycket nytta i att uppdatera de utan några större förbättringar på webbplatsen. 
\\
\\
Eftersom webbplatsen är byggd i React består den av ett antal komponenter. De tre huvudsakliga komponenterna i systemet är Topbar, Sidebar och Workspace. Topbar utgör ytan på toppen av webbplatsen och den innehåller Pelagias logga, projekttabbarna för öppna projekt och lite information om det öppna projektet. Sidebar är den yta till vänster på webbplatsen och det är där Pelagia kan bläddra bland alla existerande projekt. Projekten är där indelade i mappar som de själva kan skapa, döpa och flytta omkring. Till sist är Workspace ytan där arbete faktiskt görs. I tabellen i Workspace, på mitten av webbplatsen, skriver Pelagia in all data de hittar från prover de har analyserat.
\\
\\
I utvecklingssteget lades väldigt mycket fokus på de nyckelfunktioner som Pelagia efterfrågade från intervjuerna och presentationen av prototyperna. Det vill säga, sätt att ge de en bättre överblick av systemet och en större känsla av kontroll, samt att göra det mer effektivt och lättlärt. Det är till stor del därför systemet som byggts ser ut som det gör. För att Pelagia ens skulle kunna få en bättre överblick behövdes en helt ny struktur på hur projekten lagras (Sidebar). Dessutom, för att undvika misstag gjordes också det aktiva valet att visa enskilda projekt i tabbar så att det inte går att redigera information i fel projekt. Ytterligare, för att ge Pelagia den kontroll de ville ha i systemet utvecklades sätt att hitta projekt och prover enklare genom filtrering, sökning och flaggning. Detta gäller både för att hitta projekt, men också prover i projekt.
\\
\\
Trots att det inte var en prioritet för Pelagia var också utseendet viktigt för webbplatsen. Inte för att den måste vara överdesignad eller ens snygg för att vara bra, utan för att detaljerna i utseendet hjälper mycket med användarvänlighet. Exempel på detta är färger på knappar, olika effekter vid hover-events och andra typer av feedback vilket gör upplevelsen på sidan, medvetet eller inte, bättre. Det är just detta som har spelat den största rollen i att göra systemet mer lättlärt jämfört med Excel.
\\
\\
Under utvecklingssteget som är beskrivet ovan gjordes två designiterationer till. Den feedback som gavs från Pelagia från uppvisningen av det som byggts under dessa var dock så pass bra att inga större avvikelser gjordes från den originella planen som togs fram efter intervjuerna och feedbacken från prototyperna.