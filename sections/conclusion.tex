\section{Slutsatser}
I vårt arbete med projektet har vi dragit flera viktiga slutsatser. Både sett till designprocessen men också det tekniska genomförandet. DBT-metoden som kursen bygger på har gett oss ett viktigt verktyg för att förstå problem och utforska lösningar för att sedan kunna skapa något konkret som täcker de identifierade behoven. Att arbeta med en verklig beställare har varit mycket roligt och projektet känns som en givande erfarenhet till arbetslivet. En sorts introduktion in i hur riktiga UX-designers jobbar och den tillhörande arbetsprocessen. Med denna kurs har vi också fått insikten i hur viktigt det är med kontinuerlig dialog med beställaren så att man hela tiden kan anpassa lösningen och inte förbise beställarens behov.
\\
\\
Rent tekniskt valde vi att bygga vårt system i React, eftersom vi haft viss erfarenhet av det från tidigare kurser. Denna grundförståelse kom till användning när vi skulle skapa projektet men eftersom ingen av oss har skapat något i denna utsträckning tidigare så blev mycket av programmeringen i React ett bra övningstillfälle. Vi drar slutsatsen att React är ett fantastiskt ramverk för att skapa interaktiva gränssnitt. Vi drar även slutsatsen att god strukturering av React-filer och komponenter är avgörande för att minimera förvirring. Vi hade periodvis problem med att filer låg oorganiserat i projektet, vilket låg till orsak bakom många fel i koden och gjorde att arbetet tog längre tid. Hade vi gjort om samma projekt igen hade vi sett till att strukturera upp våra filer och komponenter bättre, direkt från start.
\\
\\
Vi har även insett vikten av att använda sig av backend, särskilt när det kommer till ett system som detta som ska användas uteslutande för lagring av data. Eftersom vi endast utvecklade frontend-delen av systemet utan någon backend kan vi inte lagra data.  
Det gör att vår implementation fungerar som ett bra bevis på konceptet, men för att systemet ska kunna driftsättas krävs en databas. Men det blir något som Pelagia behöver vidareutveckla efter projektets slut.
\\
\\
En annan viktig slutsats handlar om det gemensamma grupparbetet. Att arbeta i ett team där flera personer utvecklar samtidigt ställer krav på bra verktyg. Här upplevde vi att ett GitHub-repository var ett ändamålsenligt sådant. När projektet väl var skapat och strukturen var på plats så fungerade det stundvis utmärkt med att hämta och dela med sig av kodändringar från ett och samma ställe.
\\
\\
Så sammanfattningsvis har projektet inte bara resulterat i ett snyggt, skräddarsytt gränssnitt för Pelagia, utan också gett oss i projektgruppen praktiska lärdomar om designmetodik och hur man arbetar som konsulter.
