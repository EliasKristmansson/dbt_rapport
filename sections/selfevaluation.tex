\section{Självvärdering}
Här går vi igenom de 17 globala målen för hållbar utveckling och huruvida vårt projekt har haft en påverkan. Detta beräknas på en skala -5 till 0 till +5 där -5 är mycket negativ, 0 är ingen påverkan och +5 är mycket positiv påverkan. 
\\
\\
\textbf{Mål 1}
 - Ingen fattigdom
\\
Gradering: 0 
\\
Detta mål har vi svårt att se att vi ska ha någon påverkan på, varken positivt eller negativt. Vårt projekt underlättar vardagen för Pelagia som jobbar inom natur och biologi och hjälper externa företag och kunder, det skulle bli en extremt långsiktig relation till fattigdom. 
\\
\\
\textbf{Mål 2}
 - Ingen hunger
\\
Gradering: 0
\\
Av samma anledning som mål 1 har vårt projekt ingen nära relation till detta mål. 
\\
\\
\textbf{Mål 3}
 - God hälsa och välbefinnande
\\
Gradering: 2
\\
Pelagia har en nära relation till naturen och dess välbefinnande, både växtarter och djurliv. Att naturen mår bra och tas hand om gynnar i sin tur oss människor som vistas i den, därför går dessa hand i hand och ger oss en positiv påverkan.
\\
\\
Delmål 3.1, 3.2, 3.5, 3.7, 3.8, 3.A, 3.B, 3.C och 3.D har graderingen 0 då det är svårt att dra en direkt korrelation till vårt projekt.
\\
Delmål 3.3, 3.4, 3.6 och 3.9 har graderingen 1. De handlar om smittsamma sjukdomar, mental hälsa, skador i vägtrafiken och skadliga kemikalier och föroreningar. Alla dessa går att koppla till naturen vilket vårt företag och därigenom vårt projekt har en påverkan på. Genom att hålla natur och djur vid bra hälsa mår vi människor som vistas i den bättre, både fysiskt och psykiskt. Det kan minska trafikolyckor genom att exempelvis träd håller sig starkare och inte rasar över vägar. Dessutom vill vi skydda naturen från skadliga ämnen just för att både den ska hålla sig frisk, men även vi människor och djur som vistas och bor i den.
\\
\\
\textbf{Mål 4}
 - God utbildning för alla
\\
Gradering: 0
\\
Vi tycker inte att vårt projekt direkt har en stark nog koppling till detta för att göra en positiv påverkan. Nog handlar företaget om utbildning inom biologi och natur, men just vårt projekt gör ingen större påverkan på detta till skillnad från innan. 
\\
\\
\textbf{Mål 5}
 - Jämställdhet
\\
Gradering: 0
\\
Här har vi inte heller någon påverkan åt något håll. Företaget och projektet gynnar naturprojekt och har inte alls någon koppling till jämställdhet.
\\
\\
\textbf{Mål 6}
 - Rent vatten och sanitet för alla
\\
Gradering: 3
\\
Det här målet kan projektet och framför allt företaget som projektet hjälper ha en positiv påverkan på eftersom de jobbar med framför allt prover från olika typer av vattendrag etc. för att se till att växt- och djurliv i ekosystemet mår bra.
\\
\\
Delmål 6.2, 6.4, 6.5, 6.A har gradering 0 och har ingen nära direkt koppling till vårt projekt.
\\
Delmål 6.1, 6.3, 6.6, 6.B har gradering 3 då de handlar om säkert dricksvatten, förbättra vattenkvalitet, återställa vattenrelaterade ekosystem och lokalt engagemang i vatten- och sanitetshantering. Som beskrivet ovan jobbar företaget för att se till att allt ser bra ut i ekosystemen i naturen som proverna kommer in från vilket samtidigt bidrar till att kvaliteten på vattnet hålls uppe. 
\\
\\
\textbf{Mål 7}
 - Hållbar energi för alla
\\
Gradering: 0
\\
Här var vi inte heller varken positiv eller negativ påverkan. Varken företaget eller vårt projekt har en stark koppling till energi, utan snarare natur och ekosystem. 
\\
\\
\textbf{Mål 8}
 - Anständiga arbetsvillkor och ekonomisk tillväxt
\\
Gradering: 3
\\
I mål 8 är vi överens om att vi har en ganska stor positiv påverkan. En stor anledning till att Pelagia ville ha hjälp med sitt system är för att företaget växer och därför har vi i sin tur bidragit till den ekonomiska tillväxten direkt, om något på en liten skala. Mål 8 handlar dock mest om globala arbetsvillkor och ekonomisk tillväxt, på grund av detta kan vi inte säga att vi har en direkt påverkan på en majoritet av delmålen. Endast de som inkluderar nationella förhållanden.
\\
\\
Delmål 8.1 och 8.2 har gradering 5. Delmål 8.1 innebär att upprätthålla den ekonomiska tillväxten i enlighet med nationella förhållanden vilket vi definitivt har bidragit till, i synnerhet dock bara för ett företag i ett land. Delmål 8.2 handlar om högre ekonomisk produktivitet genom bland annat teknisk uppgradering och innovation, vilket vi också i högsta grad har hjälpt med.
\\
\\
Tyvärr handlar de resterande delmålen om problem som ligger för långt utanför omfattningen av projektet för att säga att vi har en märkbar påverkan på dem. Därför får samtliga av dessa en gradering 0, och den slutgiltiga graderingen av målet rundas ner till 3.
\\
\\
\textbf{Mål 9}
 - Hållbar industri, innovationer och infrastruktur
\\
Gradering: 3
\\
Mål 9 handlar om att bygga motståndskraftig infrastruktur, verka för en inkluderande och hållbar industrialisering samt att främja innovation. Kort och gott kan allt detta vara relevant för vårt projekt beroende på vilken typ av kund som hyr in Pelagia. Eftersom hållbarhet ligger i centrum för hela målet och Pelagia strävar efter samma sak blir det väldigt passande. Det man måste tänka på, precis som beskrivit under mål 11 också, är att det inte alltid är alla delmål som alltid är relevanta, utan det beror som sagt på vad kunden som hyr in Pelagia arbetar med.
\\
\\
Detta mål är dock något mer relevant än mål 11 eftersom infrastruktur är mer helhetstäckande är städer och samhälle. Det är nog stor sannolikhet att infrastrukturprojekt är den ledande anledningen till att kunder hyr in Pelagia. Därför för mål 9 en gradering på 3 och mål 11 en gradering på 2.
\textbf{Mål 10}
 - Minskad ojämlikhet
\\
Gradering: 0
\\
Detta mål har vi ingen påverkan på. Det vi har byggt åt Pelagia varken minskar eller ökar ojämlikheten på något sätt, varken mellan länder, nationellt eller inom företaget i sig.
\\
\\
\textbf{Mål 11}
 - Hållbara städer och samhällen
\\
Gradering: 2
\\
Mål 11 handlar om att göra städer och samhällen mer inkluderande, säkra, motståndskraftiga och hållbara. Hållbarhetsaspekten är absolut relevant för vårt projekt och främst Pelagia som företag. Eftersom Pelagia har kunder som ibland har som mål att säkerställa att den verksamhet de driver, vilket kan ha med städer och samhällen att göra, utförs på ett hållbart sätt så har vi definitivt en indirekt koppling till detta mål.
\\
\\
Vi skulle vilja argumentera att samtliga delmål under mål 11 kan vara relevanta beroende på vilket område företaget som hyr in Pelagia arbetar med. Det är därför svårt att gradera målen individuellt. På grund av detta tycker vi att det är rimligt att sätta en tvåa på samtliga delmål, vilket går att se som en kompromiss. Dels för att alla delmål i målet inte alltid är garanterade att vara relevanta, men också för att vårt projekt inte går att koppla direkt till delmålen, utan snarare Pelagias verksamhet.