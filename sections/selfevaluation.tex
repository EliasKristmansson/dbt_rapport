\section{Självvärdering}
Här går vi igenom de 17 globala målen för hållbar utveckling och huruvida vårt projekt har haft en påverkan. Detta beräknas på en skala -5 till 0 till +5 där -5 är mycket negativ, 0 är ingen påverkan och +5 är mycket positiv påverkan. 
\\
\\
\textbf{Mål 1}
 - Ingen fattigdom
\\
Gradering: 0 
\\
Detta mål har vi svårt att se att vi ska ha någon påverkan på, varken positivt eller negativt. Vårt projekt underlättar vardagen för Pelagia som jobbar inom natur och biologi och hjälper externa företag och kunder, det skulle bli en extremt långsiktig relation till fattigdom. 
\\
\\
\textbf{Mål 2}
 - Ingen hunger
\\
Gradering: 0
\\
Av samma anledning som mål 1 har vårt projekt ingen nära relation till detta mål. 
\\
\\
\textbf{Mål 3}
 - God hälsa och välbefinnande
\\
Gradering: 2
\\
Pelagia har en nära relation till naturen och dess välbefinnande, både växtarter och djurliv. Att naturen mår bra och tas hand om gynnar i sin tur oss människor som vistas i den, därför går dessa hand i hand och ger oss en positiv påverkan.
\\
\\
Delmål 3.1, 3.2, 3.5, 3.7, 3.8, 3.A, 3.B, 3.C och 3.D har graderingen 0 då det är svårt att dra en direkt korrelation till vårt projekt.
\\
Delmål 3.3, 3.4, 3.6 och 3.9 har graderingen 1. De handlar om smittsamma sjukdomar, mental hälsa, skador i vägtrafiken och skadliga kemikalier och föroreningar. Alla dessa går att koppla till naturen vilket vårt företag och därigenom vårt projekt har en påverkan på. Genom att hålla natur och djur vid bra hälsa mår vi människor som vistas i den bättre, både fysiskt och psykiskt. Det kan minska trafikolyckor genom att exempelvis träd håller sig starkare och inte rasar över vägar. Dessutom vill vi skydda naturen från skadliga ämnen just för att både den ska hålla sig frisk, men även vi människor och djur som vistas och bor i den.
\\
\\
\textbf{Mål 4}
 - God utbildning för alla
\\
Gradering: 0
\\
Vi tycker inte att vårt projekt direkt har en stark nog koppling till detta för att göra en positiv påverkan. Nog handlar företaget om utbildning inom biologi och natur, men just vårt projekt gör ingen större påverkan på detta till skillnad från innan. 
\\
\\
\textbf{Mål 5}
 - Jämställdhet
\\
Gradering: 0
\\
Här har vi inte heller någon påverkan åt något håll. Företaget och projektet gynnar naturprojekt och har inte alls någon koppling till jämställdhet.
\\
\\
\textbf{Mål 6}
 - Rent vatten och sanitet för alla
\\
Gradering: 3
\\
Det här målet kan projektet och framför allt företaget som projektet hjälper ha en positiv påverkan på eftersom de jobbar med framför allt prover från olika typer av vattendrag etc. för att se till att växt- och djurliv i ekosystemet mår bra.
\\
\\
Delmål 6.2, 6.4, 6.5, 6.A har gradering 0 och har ingen nära direkt koppling till vårt projekt.
\\
Delmål 6.1, 6.3, 6.6, 6.B har gradering 3 då de handlar om säkert dricksvatten, förbättra vattenkvalitet, återställa vattenrelaterade ekosystem och lokalt engagemang i vatten- och sanitetshantering. Som beskrivet ovan jobbar företaget för att se till att allt ser bra ut i ekosystemen i naturen som proverna kommer in från vilket samtidigt bidrar till att kvaliteten på vattnet hålls uppe. 
\\
\\
\textbf{Mål 7}
 - Hållbar energi för alla
\\
Gradering: 0
\\
Här var vi inte heller varken positiv eller negativ påverkan. Varken företaget eller vårt projekt har en stark koppling till energi, utan snarare natur och ekosystem. 
\\
\\